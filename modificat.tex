\documentclass{romjist}
\usepackage{times}
\usepackage{authblk}
\usepackage{graphicx} 
\usepackage{amsmath}
\usepackage{amssymb}
\usepackage{amsthm}
\usepackage{cite}
\usepackage{graphicx}
\usepackage{epstopdf}
\usepackage{amssymb}
\usepackage{chemfig}
\usepackage{centernot} 
\usepackage{indentfirst}
\setlength{\parindent}{1cm}

\begin{document}
      
    % the following 4 statements are for final print editing
    %do not modify them yourself
    \issuevolume{1}%set volume
    \issuenumber{1}%set number
    \issueyear{2018}%set year
    \issuepages{1--8}%set pages

    \headauthors{A. Bogdan}%short author string, for placement in heading
    \headtitle{OEEUATR}%short title string, for placement in heading
    
    % Article top matter (title, authors)
    \title{Outdoor environment exploration using autonomous terrestrial robots}%full title string

    \author[1]{Bogdan Cristian Ardeleanu}%set name of first author and superscript of affiliation mark
    \affil[1]{UndaTech - ACES S2}%affiliation for superscript mark
    \affil[ ]{Email: bogdan@unda.tech}%optionally, set email underneath affiliation, leave mark blank
    
   
    % etc, as many authors as required

    % always issue maketitle before abstract
    \maketitle
    
    \begin{abstract} After an overview of the field that introduces some of its fundamental concepts, the paper presents background material on hardware, control software architecture, and robot intelligence. It then examines a broad range of implementations and applications, including locomotion (wheeled, legged, flying, and crawling robots), localization, navigation, and mapping. The many case studies and specific applications include robots built for research, industry, and the military, among them, walking machines with four, six, and eight legs, and the. The paper concludes with recommendation on the future robot designing.\end{abstract}
    
    \begin{keywords} Mobile Robots; Mars; Joints; Shock Absorbers; Actuators; Tracking; Service Robots; Mobile Robots; Control Systems; Robot Sensing Systems; Artificial Neural Networks; Tracking; Robots; Engines; Robot Sensing Systems; Optimal Control; Observers; Legged Locomotion; Couplings; DC Motors; Stepper Motors; Servo Motors; Mobile Robots; Springs. \end{keywords}

\section{Introduction}
	In our days, robots are the masterpiece of engineering. To design and construct one you need high end electrical, mechanical and chemical engineering.
A robot is a complex machine designed with the purpose to execute one or more task automatically very fast and with good accuracy. The term comes from a Czech word, robota, meaning “forced labor”. There are as many different types of robots as there are tasks for them to perform.\par
	In this paper we will approach terrestrial robots used for environment exploration. The main reason of developing such machines is scouting dangerous or unreachable zones like volcano eruption areas, highly irradiated zones, an urban territory affected by an earthquake, caves or crevasses.\par
As a first step we will analyze the main types of motion and propulsion. Next, we will discuss about the power supply. Another important topic is environment mapping and positioning and finally, the path planning.
\section{List of papers selected by titles}
TIGRE - An autonomous ground robot for outdoor exploration; IEEE Xplore; \par\noindent Alfredo Martins, Guilherme Amaral, Andre Dias, Carlos Almeida, Jos`e Almeida, Eduardo Silva.\par
Design and implementation of a lower-limb mobile training robot; IEEE Xplore;\par\noindent Syh-Shiuh Yeh, Hung-Hsiu Yu.\par
Evolution of locomotion controllers for snake robots; IEEE Xplore;\par\noindent Janzaib Masood, Abdul Samad, Zulkafil Abbas, Latif Khan.\par
Estimation of external forces acting on the legs of a quadruped robot using two nonlinear disturbance observers; IEEE Xplore;\par\noindent Navid Dini, Vahid Johari Majd, Farid Edrisi, Mehran Attar.\par
An optimal force distribution algorithm for the legs of a hexapod walking robot; IEEE Xplore; \par\noindent C Chen. \par
Development of robot legs inspired by bi-articular muscle-tendon complex of cats; IEEE Xplore; \par\noindent Ryuki Sato, Ichiro Miyamoto, Keigo Sato, Aiguo Ming, Makoto Shimojo.\par 
Planetary soil classification based on the analysis of the interaction with deformable terrain of a wheel-legged robot; IEEE Xplore;\par\noindent Francisco Comin, Chakravarthini Saaj.\par
Towards active actuated natural walking humanoid robot legs; IEEE Xplore;\par\noindent R. C. Luo, Chwan Hsen Chen, Yi Hao Pu, Jia Rong Chang.\par
Design and control of a tracked robot for search and rescue in nuclear power plant; IEEE Xplore;\par\noindent Pengfei Dong, Xuezhu Wang, Hongjun Xing, Yiqun Liu, Meiling Zhang.\par
Simulation study of steering control of the tracked robot based on slip and skid condition; IEEE Xplore;\par\noindent Chuanwei Wang, Hongwei Ma, Kun Mam, Wanfeng Shang.\par
Tracked Robot over a Slope Path: Dynamic Stability Control; IEEE Xplore;\par\noindent Daniela D'Auria, Gianluca Ristorto, Gallo Raimondo, Fabrizio Mazzettod.\par
Learning Ground Traversability from Simulations; IEEE Xplore;\par\noindent R. Omar Chavez-Garcia, Jerome Guzzi, Luca Gambardella, Alessandro Giusti.\par
Geometric and visual terrain classification for autonomous mobile navigation; IEEE Xplore;\par\noindent Fabian Schilling, Xi Chen, John Folkesson, Patric Jensfelt.\par
Ensemble Learning With Weak Classifiers for Fast and Reliable Unknown Terrain Classification Using Mobile Robots; IEEE Xplore;\par\noindent Ayan Dutta, Prithviraj Dasgupta.\par
Acoustics based terrain classification for legged robots; IEEE Xplore;\par\noindent Joshua Christie, Navinda Kottege.\par
Obstacle detection and avoidance via cascade classifier for wheeled mobile robot; IEEE Xplore;\par\noindent Chung-Jung Lee, Teng-Hui Tseng, Bo-Jhen Huang, Jun-Weihsieh, Chun-Ming Tsai.



\section{Summaries}
\subsection{TIGRE - An autonomous ground robot for outdoor exploration}
\cite{1}The paper presents exactly an outdoor environment exploration terrestrial autonomous robot. It is developed as a versatile robotics platform for further development. TIGRE is an modular all-rounder with a lot of different sensors on the chassis of an all-terrain vehicle controlled by a general purpose microprocessor. This kind of robot is optimal for both civilian and military applications because he can execute a lot of different tasks like exploration and mapping, reconnaissance, patrolling, surveillance, establishing communications links or search and rescue. \par
A closer look at the hardware is needed to understand why this machine is so versatile. The traction is achieved through a brushless DC motor, witch produce a good torque, enabling the robot to have traction even on the roughest terrain. A set of custom made low level vehicle control subsystem like power control, direction control and traction control are connected in a CAN bus. TIGRE is powered by a set of four FiFePO4 batteries, providing a minimum 4 hours of autonomy. All processing is made by an Intel i5 based single board computer. It features 5GHz wireless communication, infrared pan and tilt thermo camera, laser rangefinder, visible spectrum camera pair for in depth vision, inertial sensors and a precision GPS receivers.\par      
Software-wise the architecture follows a modular and hierarchic structure. Lower level modules provide interface with the sensors and actuators. A multicamera target detection is implemented to produce relevant, easy to interpret, complex 3D target candidates. This images are corelated with the laser rangefinder to determine it the terrain is suitable for the vehicle or not. This information is used both in the guidance control and in the vehicle safety module for emergency stops.     
\subsection{Evolution of locomotion controllers for snake robots}
\cite{2}Evolution of artificial neural networks helps in designing locomotion controllers for robots with complex morphologies using very little a-priori knowledge. Evolution is a really promising tool for optimization in complex search spaces where human intuition is insufficient to solve the problem.\par
Snake robots, inspired by biological snakes, have the potential of navigating in cluttered environments and uneven terrains, where common wheeled or legged mobile robots cannot traverse. Snake robots have relatively higher degrees of freedom, consisting of serially connected body parts capable of bending in one or more planes.\par
We managed to design locomotion controllers for two behaviors namely, sidewinding and turning gait. Sidewinding gait makes the robot able to move sideways. The turning gait allows the robot to turn around in its place.\par
The robot was allowed to learn the behavior of sidewinding for 200 generations and this process was repeated 10 times. The same process was repeated for the 100 generations for turning behavior. 
\subsection{An optimal force distribution algorithm for the legs of a hexapod walking robot}
\cite{3}This paper presents a model for a hexapod robot to solve the optimal force distribution problem with equality and inequality constraints. To find a fast way to obtain a suitable force distribution for the hexapod robots, an analytical force distribution method using the principle of linear interpolation of optimal solution for the control is introduced. The smoothness and continuousness of the solution and its ability to prevent leg slippage is evaluated by simulation.\par
Hexapod robot, as a part of legged vehicles, broadly recognized for superior performance in its potential to be an effective and efficient transportation device on irregular or difficult terrains. Compared with those by walking only gaits based on geometric algorithms, force-controlled legged robots possess their intrinsic advantages in terms of robots' static stability, avoidance of foot slippage and optimization of joint load distribution. In these approaches, the calculation of force distribution on the robot's legs is indeed needed.\par
As known to all, statically stable walking requires at least three feet supporting on the ground. In recent years, many researchers have studied this problem and developed different algorithms for the optimal solutions. Most of them proposed specific objective functions for the problem first, and then applying various linear or quadratic programming techniques to solve it[3,4]. However, most of these methods were often too slow for real time computation and thus limited in many ways.\par
The force distribution method explored in this article is a new optimization method for a hexapod walking robot to find a suboptimal solution. The horizontal components of foot forces are considered first when calculating the solution to prevent leg slippage occurred. The principle of the linear interpolation is used in this method. So the force distribution solved by the new optimization method is continuous during a complete walking cycle.
\subsection{Development of robot legs inspired by bi-articular muscle-tendon complex of cats}
\cite{4}Within the limbs of typical animals, there exist bi-articular muscles crossing two joints. It is known that the bi-articular muscles of the felid play an important role in the locomotion. Also the muscle-tendon complex, composed of the gastrocnemius muscle and the Achilles’ tendon that cross the knee joint and the ankle joint contributes much to the movements such as running and jumping particularly. Besides, because the muscle-tendon complex has the function for absorbing shocking, it is utilized for soft landing from high places. To achieve high performance for jumping and landing motion like cats, we are developing robot legs inspired by the bi-articular muscle-tendon complex of cats. The leg consists of hip, knee and ankle joints. For the knee and ankle joints, a four-bar linkage mechanism with one elastic linkage, in which the knee joint is driven by an electric rotary motor and the ankle joint is passive, is applied. By this mechanism, basic functions of the bi-articular muscle-tendon complex of felids like cats can be realized and the performance for jumping and landing can be improved. In this paper, the new leg mechanism is described. Moreover, a prototype of a pair of the hind legs of the quadruped robot using the new mechanism has been developed. The results of jumping and landing experiments are shown to validate the effectiveness of the mechanism.\par
This muscle-tendon complex is extended when animals touchdown and most part of the kinetic energy can be stored in the complex. Many animals utilize the stored energy for next takeoff motion efficiently. In fact, the kangaroo can perform consecutive hopping in high efficiency with the small muscle due to the utilization of the elastic energy stored in the tendon. By referring the complex, a leg mechanism shown in Fig.1(b) is proposed. The mechanism is composed of active hip and knee joints driven by electric rotary motors and passive ankle joint. Knee joints and ankle joints are connected by a four bar linkage mechanism in which one linkage is elastic. The elastic four-bar linkage mechanism hereafter is referred to as EFLM. The elastic link is realized by extension spring connected between the thigh linkage and foot linkage. And the length of the elastic linkage is limited to a certain range. The minimum length corresponds to the position where the spring is fully extracted. The maximum length corresponds to the position where the EFLM is a parallel mechanism. The EFLM is to realize the function of the bi-articular muscle-tendon complex
\subsection{Towards active actuated natural walking humanoid robot legs}
\cite{5}The objective of this paper is to investigate towards active actuated natural walking humanoid robot legs. Conventional humanoid robots suffer from problems like artificial and unnatural motion, or low agility. To improve the performance of the humanoid robot, this paper introduces the idea which employs the active-actuated biped robot legs and the passive dynamic walkers with more naturally walking. The approach is primarily based on the utilization of shock absorber and parallel linkage mechanism and the walking algorithm which combines “Series Elastic Actuation” and “Limit Cycle Walking”. The shock absorbers mounted in series way enable the implementation of the conventional “Series Elastic Actuation”, and the shock absorbers mounted in parallel way are added as the modification in order to store the energy of the actuators like the human muscle. The specifically designed mechanism provides better support and lower load of the motor than traditional design. The hardware prototype has been implemented. The simulation and analysis demonstrate the high potential and possibilities of this concept and provide the new direction towards the naturally walking humanoid robot legs design.\par
Our goal is to exploit both the benefit of the active actuated robots and the passive dynamic walkers, while active-actuated robots tend to have high versatility and thus are able to perform multiple tasks, and passive dynamic walkers have better performance in terms of efficiency and walk more human-like and natural.\par
The requirement of our robot is to walk and turn smoothly as well as maintain high lateral stability simultaneously. Therefore the robot should possess all the functions of the conventional humanoid walking robot. Each of the robot’s leg has seven degrees of freedom, active joints include roll, and yaw joints of the hip. Considered that the lateral stability is mainly maintained by the roll joint of hip, the robot possesses only a passive roll joint at ankle. A passive movable ”toe” mechanism is also designed to make the robot walk more human-like, and able to perform toe-off motion.\par
The specifically designed mechanism for humanoid robot leg is proposed in this paper. It features the parallel and series shock absorbers and parallel linkage mechanism design. The stretching and compression motion of linear actuator and the shock absorber completely mimic the function of human muscle, along with the parallel linkage mechanism, the energy efficiency is also elevated to a level closer to human muscle. The compliant mechanism with shock absorbers also wisely adopt the concept of suspension from street vehicle and thus has the ability to support the whole structure that weighs much more than the maximum output force of actuators. In other words, the lighter actuators of lower output force can be chosen, correspond to the aforementioned idea of higher efficiency. Furthermore, the mechanism also offers more capability to tolerate the impact comes from ground.
\subsection{Design and control of a tracked robot for search and rescue in nuclear power plant}
\cite{6}Environment in some special areas of nuclear power plant is very bad, and is even worse and inaccessible after a nuclear accident. Therefore, a robot developed for special occasions of nuclear plant operation is urgently needed. Different from ordinary robots, nuclear industry robots face even worse environments, such as high radiation, high temperature, high pressure, narrow space, slopes, stairs, and complicated pipeline. This paper designs a tracked mobile robot for search and rescue in nuclear plant. Stair climbing performance of the robot is analyzed. Hardware and software of the control system are designed, and the human-computer interaction interface is developed. Dynamic model of the robot is established, and the trajectory tracking control algorithm of the robot is presented. Finally, simulation experiments using Octave software are finished. Simulation results show that the proposed control algorithm could enable the tracked robot to achieve excellent trajectory tracking of linear and cambered path, and the effectiveness of the proposed method is verified.\par
Ground locomotion types include wheeled, tracked, legged, peristaltic and so on. Wheeled locomotion has high speed, but only applicable to flat ground. Legged locomotion has superior terrain adaptability, but its convenience of control and reliability in nuclear plant is inadequate, such as robot SILO6 and COMET-IV. Therefore, tracked robot has appropriate terrain adaptability and technology maturity, and is the best choice for search and rescue in nuclear plant. Many studies on tracked robots had been carried out over the world.\par
The tracked robot is made up of four tracks and body. The robot body is mainly composed of chassis, cover and diaphragm. The manufacture of the robot adopts seamless welding technology to ensure firm structure, accurate assembling and good waterproof performance. The battery cabin is symmetrically arranged on both sides of the chassis to provide power support for robot moving. The middle part of robot body is set with a detachable diaphragm to divide the interior space into two parts, the first one installs motor, driver and other large power devices, and the second one installs computer and wireless communication devices. This design helps to prevent electromagnetic interference caused by large power devices to control system. The cover is detachable, and the upper part is provided with a mechanical interface for monitoring camera and sensing system. The cover and the chassis are connected by bolts, and the interface is sealed by sealing strip. The front and rear of the body install forward-looking camera and rearview camera respectively, and a panoramic camera is installed in the upper part of the body. The power switch, wireless communication antenna, charging interface, wired control interface and extended interface are all arranged at the rear of the body.\par
In order to improve the obstacle crossing ability of the robot, there are four tracks in use, including two main tracks and a pair of swing arm tracks. The inner surface of the track is provided with supporting wheels to prevent its deformation and prolong its life span. The two main tracks install on both sides of the body, each consists of a circular track, a driving wheel, a guide pulley, and several supporting wheels. Rotation of swing arm is driven by separate motors. Two other motors actuate the two main tracks on both sides of the body to achieve forward and backward moving, differential steering can be realized by controlling the speed and direction of the two motors to achieve fast steering and turning with zero radius.
\subsection{Simulation study of steering control of the tracked robot based on slip and skid condition}
\cite{7}The steering characteristic of tracked robot was mastered by analyzing its steering principle. Under certain road conditions, the experiments of accurate steering control were finished by modifying the velocity parameters of two tracks. The relationship is gained between the speed of robot's tracks and the steering motion of the robot through analyzing robot's steering-kinematic principles. Under the influence of slip and skid, the tracked robot's relativistic steering radius under actually conditions is larger than in ideal conditions. At last, the correctness of the tracked robot's steering control theory was verifies by simulation experiments of the steering movements along the line of its edge trimming.\par
Tracked robot's driving dynamics has been deep studied by many scientific research institutions at home and abroad, since it has strong capable at obstacle surmounting performance, especially in the driving performance of the tracked robot. Wong has established the models of interaction effect about track, wheels and the ground, and compared the movement performance of tracked vehicles with the off-road vehicle which has several wheels. In order to establish accurate steering models of the tracked vehicles and improve its steering performance, the steering-kinematic principles based on slip and skid is used widely and simulation study is made to the scheme with relevant software. Intelligence control of tracked mobile robot is often studied by visual, ultrasonic or infrared, and without considering the driving dynamics of tracked robot. The steering performance of the tracked robot and the steering control strategy will be studied, and the result has important directive significance to tracked robot's precise control and accurate positioning.\par
There are several similarities in the steering principle between tracked robot and tank, but the pivot steering of tank is achieved through controlling track's speed which is equal in magnitude and opposite in direction, and tracked vehicle steers on the ground by driving one track while braking on the other track, but the tracked robot has several ways to fulfill steering movements.
\section{Conclusions}
As a bigger picture we can say that there is no perfect robot, viewed as a complex system, neither   perfect sub-system but we can choose the best one for the needed application. Even if we want to design a robot for a specific task, there are a lot of variables. We can choose the optimal one. For that we need a good design.\par
A Robotic design is the creation of a plan for the construction of a robot or a robotic system. Design has different connotations in different fields. More formally a robotic design is defined as; (noun) a specification of a robot, manifested by a robot designer, intended to accomplish goals, in a particular robotic environment, using a set of primitive components, satisfying a set of requirements, subject to constraints; (verb, transitive) to create a robotic design.\par
You need to determine what problem you are trying to solve and the objectives you want to reach before you attempt to design and build a robot. Take the time to study a number of different situations and once you have decided what the situation is and you understand exactly what the problem is then write a design brief in a log book. Many times, robot designers, and engineers do not dream up an idea on their own, but are bombarded by the problems of a customer, society, or the environment that has to be solved to achieve a basic “need.” Without a clear definition of this need and the objectives, the engineering design process cannot begin. Much time and many careers have been wasted in the pursuit of an un-defined target. Get a clear picture of the parameters of the problem. Make a list of the objectives and rank them in order of importance. Define the constraints of the problem. Many times a robot cannot do everything that a problem presents. It is important to prioritize and design a machine that can do the most things and do a few things very well.\par
Having written a brief, you are now ready to gather information. First you will need to decide what information you require. This will be different from project to project and will also depend on the amount of information and knowledge you already have. A design’s practical functions can include: movement, energy, intelligence, sensing. Research must be focused and incorporate new ideas and a thorough exploration of old similar ideas. Sometimes the old ideas are the best. Ever heard the saying, “Don’t reinvent the wheel?” Old ideas that failed are sometimes great research gold mines; that idea may have failed due to a lack of new technology that may exist now.\par
The best way to know if a design will work in real-world conditions is to build a prototype. Sketches and notes are required at this stage. Print out the picture and jot your notes below the picture in your log book. Once you have settled on a solution, go back over the list of specifications you have made. Make sure that each specification is satisfied. If an initial design and prototype does not fully solve the problem or specifications, meet the design parameters, or stay within an acceptable cost, a designer may go “back to the drawing board” (or computer). The engineering design process has a loop to go back to the design and refine or redesign. Now it is the time to produce some working drawings. These are the drawings that will assist you as you begin constructing your robot. The build process must take into consideration materials, processes, construction limitations, and cost. Companies make substantial investments in factories and the infrastructure to build their designs so the more efficiently a design has been handled, the better off the build will be. Once the build process has begun, the company can begin to hopefully make a return on its investments in the entire design process by marketing and selling the product.



\begin{thebibliography}{24}
\bibitem{1}	Alfredo Martins, Guilherme Amaral, Andre Dias, Carlos Almeida, Jos`e Almeida, Eduardo Silva. \textit{TIGRE - An autonomous ground robot for outdoor exploration}; IEEE Xplore; 
\bibitem{2}   Janzaib Masood, Abdul Samad, Zulkafil Abbas, Latif Khan. \textit{Evolution of locomotion controllers for snake robots}; IEEE Xplore;
\bibitem{3}	C Chen. \textit{An optimal force distribution algorithm for the legs of a hexapod walking robot}; IEEE Xplore; 
\bibitem{4}	Ryuki Sato, Ichiro Miyamoto, Keigo Sato, Aiguo Ming, Makoto Shimojo. \textit{Development of robot legs inspired by bi-articular muscle-tendon complex of cats}; IEEE Xplore; 
\bibitem{5}	R. C. Luo, Chwan Hsen Chen, Yi Hao Pu, Jia Rong Chang. \textit{Towards active actuated natural walking humanoid robot legs}; IEEE Xplore
\bibitem{6}	Pengfei Dong, Xuezhu Wang, Hongjun Xing, Yiqun Liu, Meiling Zhang. \textit{Design and control of a tracked robot for search and rescue in nuclear power plant}; IEEE Xplore
\bibitem{7}	Chuanwei Wang, Hongwei Ma, Kun Mam, Wanfeng Shang. \textit{Simulation study of steering control of the tracked robot based on slip and skid condition}; IEEE Xplore;
\end{thebibliography}

\end{document}
